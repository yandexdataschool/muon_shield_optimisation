\section{Project plan}
\label{sec:projectplan}

\begin{itemize}
    \item Project plan in synch with BDF plan
    \item Extra year of stop of SPS (last year of LS3)
    \item commission of facility and experiment after LS3
    \item prototyping plan (reference to document - update?)
    \item COSTING detector, rough statements about impact of changes and expectation for total cost, no detailed breakdown if any at all
    \item safety file and planning
\end{itemize}

The detector R&D schedule, TDR, PRR, production.

The schedule is elaborated in more detail in the BDF Yellow Book~\cite{ref:bdf_yellowbook}

The schedule for the SHiP experiment and the experimental facility is largely driven by the CERN long-term accelerator schedule. Accordingly, the schedule aims at profiting as much as possible from data taking during Run 4 (currently 
2027--2029). 

Most of the experimental facility can be constructed in parallel to operating the North Area beam facilities. The connection to the SPS has been linked to Long Shutdown 3 (i.e. for LHC 2024--2026) but requires that the stop of the North Area is extended by one year (2025--2026). The schedule requires preparation of final prototypes and the TDRs for both the detector and the facility by beginning 2022, and construction and installation between 2023 and beginning 2027.  

% The second half of the dedicated transfer line, the target area and the experimental area can be 
% constructed in parallel to operating the current North Area beam facilities. In order to minimize the
% impact on the North Area, the connection to the SPS requiring the construction of a junction cavern 
% and the installation of the new splitter and beam line, has been linked to Long Shutdown 3. Since the 
% time required for this work is estimated to 2 years, the stop of the North Area must be extended by 
% one year (2025–2026).


